\section{Introduction}
Systems/services used to perform critical functions require justifications that they exhibit necessary properties (i.e. safety and/or security). 
\textit{Assurance case}s provide an explicit means for justifying and assessing confidence in these critical properties. 
In certain industries, typically in the safety-critical domain (e.g. defence, avaiation, automotive and nuclear), it is a regulatory requirement that an assurance case is developed and reviewed as part of the certification process \cite{healthFound}.
An assurance case report is a document that facilitates information exchange between various system stakeholders (e.g. between operator and regulator), where the knowledge related to the safety and security of the system is communicated in a clear and defendable way \cite{hawkins2013assurance}. 

Assurance case reports are typically represented textually - using natural languages; or graphically - using structured graphical notations such as the Goal Structuring Notation (GSN) \cite{kelly2004goal} or Claims, Arguments and Evidence (CAE) \cite{cae}. 
A number of tools exist which implement GSN and CAE to produce argumentation diagrams. 
Some tools adopt model-based approach to produce models that conform to their own versions of GSN and CAE metamodels. 

%A number of tools have been implemented to create GSN and CAE diagrams to ease the creation of graphical safety and/or security argumentations. Amongst the tools, only a few tools provide a model-based solution. In addition, model-based tools often lack the integration with and the traceability to, design artefacts, which pose a significant challenge to the development and acceptance of assurance cases.
%The creation of assurance cases is often a manual process, there have been some approaches and tools to apply principals of Model Driven Engineering (MDE) in order to benefit from automation provided by MDE. 

To improve standardisation and interoperability, the Object Management Group (OMG) specified and issued a Structured Assurance Case Metamodel (SACM), which can be used to construct model-based assurance cases. 
SACM is more powerful than GSN and CAE in the sense that it captures not only the argumentation regarding system properties, but also the inter-relationship between the argumentation and its supporting artefacts.
However, neither detailed explanation has been provided in the OMG specification to demonstrate how to use SACM, nor the relationships between existing assurance case approaches (e.g. GSN and CAE) and SACM has been discussed. 
This brings challenges to the adoption of SACM due to the complex structure of SACM and the sophistication of its intended usage. 

Model-based system assurance has attracted significant amount of interests in recent years due to the benefits provided by Model-Driven Engineering (MDE) such as automation, consistency and efficiency. 
With the development of Cyber-Physical Systems (CPS) and Internet of Things (IoT), model based system assurance seems particularly important where automated run-time system assurance is needed (e.g. for collaborative CPSs that form temporary system of systems at runtime). SACM provides a solid foundation for such scenarios. \will{We need to maybe talk a bit more on CPS and IoT in a later section.}


%Broadly speaking, the development and acceptance of assurance cases poses significant challenges for engineers and assessors. This is due to the size and complexity of assurance cases, where explicit and implicit dependencies exist between the argumentation for safety/security in the assurance case and its supporting evidence/artefacts, which may include system test data, system failure and hazard analysis, system architecture, etc. In particular, the lack of integration with, and the limited traceability to these evidence/artefacts undermine confidence in the argumentation presented in the assurance case. 

In this paper, we provide a detailed explanation of SACM and discuss its relationship with existing system assurance approaches. 
We first discuss the motivation of our work. 
We then provide detailed discussions about the facilities provided by SACM\footnote{Version 2.0 as of April, 2019}. 
%We discuss the relationship between existing system assurance approachs (e.g. GSN and CAE) and SACM. 
We demonstrate how SACM can be used to construct an assurance case using examples, and how to maintain the traceability between the argumentation of assurance cases and their supporting evidence/artefacts. 
We finally discuss briefly tool support for model-based system assurance, namely the Assurance Case Modelling Environment (ACME).

The contributions of this paper are as follows:
\begin{itemize}
	\item A definitive exposition of SACM version 2.0;
	\item A detailed discussion on how to use SACM to create a complete system assurance case.
	\item GSN and CAE metamodels that are compliant with SACM;
	\item Comprehensive mappings from GSN/CAE to SACM;
\end{itemize}

This paper is organised as follows. 
In Section~\ref{sec:background} and Section~\ref{sec:gsn}, we provide the motivation and the background of our work. 
In Section~\ref{sec:sacm} we provide detailed discussions about SACM. 
We also provide examples on how to use SACM to construct an assurance case. 
In Section~\ref{sec:mapping}, we discuss the relationship between existing notations and SACM. 
We provide SACM compliant metamodels for GSN and CAE and their mappings to SACM. 
In Section~\ref{sec:toolsupport}, we briefly discuss tool support for model-based system assurance. 
We finally conclude the paper in Section~\ref{sec:conclusion}.

