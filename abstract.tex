\begin{abstract}
Assurance cases are used to demonstrate confidence in properties of interest for a system (e.g. for safety and/or security). 
A number of approaches for assurance case construction are adopted by industries in the safety-critical domain. 
However, the task of constructing assurance cases remain a manual, lengthy and informal process. 
%Thus, model based approaches have been proposed to promote automation.
The Structured Assurance Case Metamodel (SACM) is a standard specified by the Object Management Group (OMG), which is designed to be used to construct assurance case models. 
However, the adoption of SACM faces challenges as the usage of it has not been sufficiently explained. 
In addition, the lack of concrete syntax (e.g. graphical syntax) of SACM makes it difficult to construct. 
More importantly, there has not been support to interoperate between existing assurance models (such as Goal Structuring Notation models) and SACM models. 

In this paper, we provide a definitive guide to SACM to explain its intended usage, and to illustrate the interoperability between existing assurance models to SACM models. 
We do so by illustrating how SACM can be used through examples. 
In addition, to promote model based approach, we provide a SACM compliant metamodel for the Goal Structuring Notation (GSN) and the transformation between GSN and SACM.
We propose a systematic approach for model based system assurance using SACM, in the sense that all corresponding information can be bridged together using the facilities provided in SACM.

\keywords{Mode Driven Engineering, Structured Assurance Case Metamodel, Model Based System Assurance, Goal Structuring Notation}
\end{abstract}
