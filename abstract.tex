\begin{abstract}
Assurance cases are used to demonstrate confidence in system properties of interest (e.g. for safety and/or security). 
A number of approaches for constructing assurance cases are adopted by industries in the safety-critical domain. 
However, the task of constructing assurance cases remain a manual, lengthy and informal process. 
%Thus, model based approaches have been proposed to promote automation.
The Structured Assurance Case Metamodel (SACM) is a standard specified by the Object Management Group (OMG). 
SACM is designed to be used to construct assurance case models which is more powerful than existing approaches. 
However, the intended usage of SACM has not been sufficiently explained. 
In addition, there has not been support to interoperate between existing assurance case models (such as Goal Structuring Notation models) and SACM models.
SACM provides a solid foundation for model-based system assurance, which bears great application potentials in growing technology domains such as autonomous systems, Cyber-Physical Systems (CPS) and Internet of Things (IoT). 

In this paper, we explain the intended usage of SACM. We also explain the relationship between existing assurance cases and SACM models. 
In addition, to promote a model based approach, we provide SACM compliant metamodels for the Goal Structuring Notation (GSN) and Claims, Arguments and Evidence (CAE),  and the transformation from GSN and CAE to SACM.
We propose a systematic approach for model based system assurance using SACM, in the sense that all corresponding information can be bridged together using the facilities provided in SACM.
We also discuss briefly the application of model-based system assurance in the context of CPS and networks of CPSs.

\keywords{Mode Driven Engineering, Structured Assurance Case Metamodel, Model Based System Assurance, Goal Structuring Notation}
\end{abstract}
