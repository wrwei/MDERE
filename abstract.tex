\begin{abstract}
Assurance cases are used to demonstrate confidence in properties of interest for a system (e.g. for safety and/or security). 
A number of approaches for assurance case construction are adopted by industries in the safety-critical domain. 
However, the task of constructing assurance cases remain a manual, lengthy and informal process. 
Thus, model based approaches have been proposed to promote automation.
The Structured Assurance Case Metamodel (SACM) is a standard specified by the Object Management Group (OMG), which is designed to be used to construct (machine-readable) assurance case models. 
However, the adoption of SACM faces challenges as there is a cognitive gap between the syntax/semantics of SACM elements and system assurance practitioners' understanding of them.

In this paper, we provide a definitive guide to SACM.
We do so by discussing the relationship between SACM and the Goal Structuring Notation (GSN), we also demonstrate how SACM can be used through examples. 
We then propose a systematic approach for model based system assurance using SACM, in the sense that all corresponding information can be bridged together using the facilities provided in SACM.


SACM is a product of 10 years efforts involving a wide variety of organisations. OMG activity, system assurance task force, initial work on ARM, structured assurance metamodel, SACM1 and SACM2

interchange format for assurance case among stakeholders

enabling tool-supported model-based approach for constructing assurance cases, including potential of automation.

modularity
patterns
controlled language

barbra paper SACM++


AIF, argument interchange format

highlight the utility of SACM being a common format.

A little bit more model transformation and validation. Transformation from CAE to SACM


Using validation to validate assurance cases, using model transformations to inteoperate between GSN/CAE to SACM

CAE metamodel 15026 part 2


issue of confidence
support in formal logic

support for the dialectic: GSN/CAE

Improved support for modularity and packaging

support for patterns

support for structured expression 

better links to artefacts


ACE 

\keywords{Mode Driven Engineering, Structured Assurance Case Metamodel, Model Based System Assurance, Goal Structuring Notation}
\end{abstract}
