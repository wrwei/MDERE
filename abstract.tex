\begin{abstract}
Assurance cases are used to demonstrate confidence in properties of interest for a system (e.g. for safety and/or security). Typically, the task of constructing assurance cases is a lengthy and informal process. 
Thus, model based approaches have been proposed to promote automation.
The Structured Assurance Case Metamodel (SACM) is an OMG specification designed to represent system assurances in (machine-readable) models, so that a series of model management operations can be performed to automate the process of system assurance. However, the adoption of SACM faces challenges as there is a cognitive gap between the syntax/semantics of SACM elements and system assurance practitioners' understanding of SACM.

In this paper, we provide a definitive guide to SACM with examples so that SACM can be better understood. We also discuss the relationship between SACM and the Goal Structuring Notation (GSN) and the interoperability between them. We also propose a systematic approach for model based system assurance using SACM, in the sense that all corresponding information can be bridged together using the facilities provided in SACM.

\keywords{Mode Driven Engineering, Structured Assurance Case Metamodel, Model Based System Assurance, Goal Structuring Notation}
\end{abstract}
