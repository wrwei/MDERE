\begin{abstract}
Assurance cases are used to demonstrate confidence in properties of interest for a system (e.g. for safety and/or security). 
A number of approaches for assurance case construction are adopted by industries in the safety-critical domain. 
However, the task of constructing assurance cases remain a manual, lengthy and informal process. 
Thus, model based approaches have been proposed to promote automation.
The Structured Assurance Case Metamodel (SACM) is a standard specified by the Object Management Group (OMG), which is designed to be used to construct (machine-readable) assurance case models. 
However, the adoption of SACM faces challenges as there is a cognitive gap between the syntax/semantics of SACM elements and system assurance practitioners' understanding of them.

In this paper, we provide a definitive guide to SACM.
We do so by discussing the relationship between SACM and the Goal Structuring Notation (GSN), we also demonstrate how SACM can be used through examples. 
We then propose a systematic approach for model based system assurance using SACM, in the sense that all corresponding information can be bridged together using the facilities provided in SACM.

\keywords{Mode Driven Engineering, Structured Assurance Case Metamodel, Model Based System Assurance, Goal Structuring Notation}
\end{abstract}
