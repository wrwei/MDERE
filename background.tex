\section{Background and Motivation}
\label{sec:background}

\subsection{Safety Cases}
\will{motivation: argumentation and graphical argumentation, no common understanding of SACM, CPS and networks of CPS} 
%System Assurance is the process of building clear, comprehensive and defensible arguments regarding the properties of systems (i.e. safety and/or security\footnote{\textit{System properties} refer to system safety and/or security hereafter}. 
The concept of assurance cases has been long established in the safety-related domain, where the term \textit{safety case} is normally used. 
For many industries, the development, review and acceptance of a safety case forms a key element of regulatory processes. This includes the nuclear \cite{hse}, defence \cite{mod2007}, civil aviation \cite{caa2007} and railway \cite{yellowBook2007} industries. Safety cases are defined in \cite{kelly2004goal} as follows: \textit{A safety case should communicate a clear, comprehensible and defensible argument that a system is acceptably safe to operate in a particular context}. 

Historically, safety arguments were most typically communicated in safety cases through free text. However, there are problems experienced when text is the only medium available for expressing complex arguments. One problem of using free text is that the language used in the text can be unclear and poorly structured, there is no guarantee that system engineers would produce safety cases with clear and well-structured language. Also, the capability of expressing cross-references for free text is very limited, multiple cross-references can also disrupt the flow of the main argument. Most importantly, the biggest problem with using free text is in ensuring that all stakeholders involved share the same understanding of the argument to development, agree and maintain the safety arguments within the safety case \cite{kelly2004goal}.

To overcome the problems of expressing safety arguments in free text, graphical argumentation notations have been developed. Graphical argumentation notations are capable of explicitly representing the elements that form a safety argument (i.e. requirements, claims, evidence and context), and the relationships between these elements (i.e. how individual requirements are supported by specific claims, how claims are supported by evidence and the assumed context that is defined for the argument). 
Amongst the graphical notations, the Goal Structuring Notation (GSN) \cite{kelly2004goal} has been widely accepted and adopted \cite{chinneck2004turning}. 
The key benefit experienced by companies/organisations adopting GSN is that it improves the comprehension of the safety argument amongst all of the key project stakeholders (e.g. system developers, safety engineers, independent assessors and certification authorities), therefore improving the quality of the debate and discussion amongst the stakeholders and reducing the time taken to reach agreements on the argument approaches being adopted.

A number of drawing tools have been developed \cite{asce, iscade, yorkgsn} to produce GSN diagrams. Although GSN diagrams produced by the majority of the tools are valuable in communicating safety argumentations amongst stakeholders, these diagrams cannot be consumed by computers to achieve higher level goals (such as automated validation of safety argumentation, automated generation of safety case reports, etc.). 

%Some tools support exporting the diagrams into machine consumable models \cite{}, however, these tools implement their own versions of the GSN metamodel.

\subsection{Safety Cases and Model Driven Engineering}
Model-Driven Engineering (MDE) is a contemporary software engineering approach. In MDE, \textit{model}s are first class artefacts, therefore drives the development. There are two important aspects in MDE: domain specific modelling and model management. Domain specific modelling enables domain experts to capture the concepts in their systems in the form of \textit{metamodel}s, which are then used to create models of their systems (that conform to the defined \textit{metamodel}s). Model management enables a series of operations to be performed on models in an automated manner, which include, but not limited to: Model Validation, Model-to-Model Transformation, Model-to-Text Transformation. MDE has been proven to improve productivity significantly due to the automation that model management provides \cite{}. 

There is a tendency for tools to adopt MDE, it is no exception for GSN tools. Some GSN tools support exporting GSN diagrams into machine consumable models \cite{}. However, these tools implement their own versions of the GSN metamodel, that do not consider the implicit reference from safety argumentations to their supporting evidence, which provides little value in performing model management operations on them.


%To promote re-use of good practice when creating structured argumentations, GSN provides a mechanism where \textit{Safety Case Pattern}s can be defined. Safety case patterns are essentially argumentation templates, which will be populated when the patterns are \textit{instantiated} \cite{kelly1997safety}. However, the pattern instantiation has often been a manual process. In \cite{}, the authors identify the need to automate safety case pattern instantiation and point out that the use of Model-Driven Engineering (MDE) can 
%

\subsection{Assurance Cases and the Structured Assurance Case Metamodel}
There have been increasing interests in the use of structured argumentation in other domains, particularly for demonstrating system security \cite{}. Such argumentations are typically referred to as \textit{security cases}. The similarities between safety and security cases have been highlighted in \cite{}. Therefore, the term \textit{assurance case} is a broader definition: \textit{An assurance case should communicate a clear, comprehensible and defensible argument that a system/service is acceptably safe and/or secure to operate in a particular context.} 

To promote standardisation and interoperability, the Object Management Group (OMG) specified and issued the \textit{Structured Assurance Case Metamodel}. In version 2.0 of SACM\footnote{Released in September, 2017}, a considerable amount of changes have been introduced to promote modularity and openness of SACM. SACM provides a complete solution where structured argumentation can be made. In addition, supporting evidence and related information can also be referenced via implicit links provided in SACM.

However, in the SACM specification there is limited information on how facilities in SACM can be used. To exploit SACM's full potential, we think it is compelling to provide a definitive guide to SACM so that safety and security engineers can fully use SACM to construct their assurance cases and perform higher level operations on them. 

Although SACM provides a complete solution to model based system assurance, in the current practice, GSN remains the most popular notation to construct assurance cases. Thus, there is a need to interoperate between GSN and SACM. In this paper, we provide a GSN metamodel (that aligns to SACM) and a model-to-model transformation which transforms GSN models to SACM models. 

\subsection{Model-based System Assurance and Cyber-Physical Systems}
\will{Need to elaborate, this is too brief}
Cyber-Physical Systems (CPS) has attracted a significant amount of research interest in recent years for its enormous potential for economic and societal impact in domains such as mobility, home automation and health care \cite{}. The open and cooperative nature of CPS poses a significant new challenge in assuring dependability. In \cite{}, the authors propose the idea of exchanging models at runtime to enable the automated reasoning of system dependability. In \cite{}, it is also proposed that models which contain assurance cases and other related information should be made available at design time, so that when integrating systems, assurance cases can be integrated. 

Therefore, model-based assurance cases are more than 'nice diagrams', they are important artefacts which can be used at design time system integration, as well as at runtime for automated reasoning when CPS collaborate with each other and form networks of CPS. 

