\section{Background and Motivation}
\label{sec:background}

\subsection{Safety Cases}
The concept of assurance cases has been long established in the safety-related domain, where the term \textit{safety case} is normally used. 
For many industries, the development, review and acceptance of a safety case forms a key element of regulatory processes. This includes the nuclear \cite{hse}, defence \cite{mod2007}, civil aviation \cite{caa2007} and railway \cite{yellowBook2007} industries. 
Safety cases are defined in \cite{kelly2004goal} as follows: \textit{A safety case should communicate a clear, comprehensible and defensible argument that a system is acceptably safe to operate in a particular context}. 

Historically, safety arguments were typically communicated in safety cases through free text. However, there are problems experienced when text is the only medium available for expressing complex arguments. 
One problem of using free text is that the language used in the text can be unclear and poorly structured, there is no guarantee that system engineers would produce safety cases with a clear and well-structured language. 
Also, the capability of expressing cross-references for free text is very limited, multiple cross-references can also disrupt the flow of the main argument. 
Most importantly, the problem with using free text is in ensuring that all stakeholders involved share the same understanding of the argument to development, agree and maintain the safety arguments within the safety case \cite{kelly2004goal}.

To overcome the problems of expressing safety arguments in free text, graphical argumentation notations were developed. 
Graphical argumentation notations are capable of explicitly representing the elements that form a safety argument (i.e. requirements, claims, evidence and context), and the relationships between these elements (i.e. how individual requirements are supported by specific claims, how claims are supported by evidence and the assumed context that is defined for the argument). 
Amongst the graphical notations, the \textit{Goal Structuring Notation} (GSN) \cite{kelly2004goal} has been widely accepted and adopted \cite{chinneck2004turning}. 
The key benefit experienced by companies/organisations adopting GSN is that it improves the comprehension of the safety argument amongst all of the key project stakeholders (e.g. system developers, safety engineers, independent assessors and certification authorities), therefore improving the quality of the debate and discussion amongst the stakeholders and reducing the time taken to reach agreements on the argument approaches being adopted.

Another popular graphical argumentation notation is \textit{Claims-Arguments-Evidence} (CAE) \cite{bishop2000methodology}. 
CAE views assurance cases as a set of \textit{Claim}s supported by \textit{Argument}s, which in turn rely on \textit{Evidence}.
Compared to CAE, GSN provides more granular decomposition of safety arguments into \textit{Goal}s, \textit{Context}s, \textit{Assumption}s, \textit{Justification}s, \textit{Strateg-}ies and \textit{Solution}s. GSN supports additional features such as \textit{Module}s, \textit{Contract Modules} and argument templates (GSN Patterns) to promote modularity and re-use.
In this paper we will focus on GSN (and its relationship to SACM) since we are the principal contributors to the standardisation of both GSN and SACM.

A recent study \cite{maksimov2018} has looked into and compared assurance case tools that have been developed in the past twenty years. 
The study identifies 37 assurance case tools, where 32 of them offer support for GSN. 
Some tools also support SACM \cite{denney2017tool, larrucea2017supporting, matsuno2010dependability, barry2011certware, netkachova2014tool}. 
%Among the tools, AdvoCATE \cite{denney2017tool}, Astah GSN \cite{larrucea2017supporting} and D-Case Editor \cite{matsuno2010dependability} support GSN and claim compliance with SACM. 
%Tools CertWare \cite{barry2011certware} and ASCE \cite{netkachova2014tool} support GSN and CAE, and also claim compliance with SACM.
However, the claimed support for SACM in fact refer to SACM version 1.0 (released in June, 2015), which was replaced by SACM version 2.0 (released in March, 2018). 

%Whilst the assurance case diagrams produced by the tools are valuable in communicating system assurance argumentations amongst stakeholders, the diagrams are mostly not model-based (except for the ones created by CertWare \cite{barry2011certware} and Astah GSN \cite{larrucea2017supporting}). 
%Thus, automated model management operations (model-to-model transformations, model-to-text transformations, model validations, model mergings, etc.) on such diagrams can not be performed on the diagrams. 

\subsection{Safety Cases and Model Driven Engineering}
Model-Driven Engineering (MDE) is a contemporary software engineering approach. 
In MDE, \textit{model}s are first class artefacts, therefore driving the development. 
There are two important aspects in MDE: domain specific modelling and model management. 
Domain specific modelling enables domain experts to capture the concepts in their systems in the form of \textit{metamodel}s, which are then used to create models of their systems (that conform to the defined \textit{metamodel}s). 
Model management enables a series of operations to be performed on models in an automated manner, which include, but not limited to: Model Validation, Model-to-Model Transformation, and Model-to-Text Transformation.
MDE has been proven to improve productivity significantly due to the automation provided by model management operations \cite{jaaksi2002developing, karna2009evaluating}. 

There are a number of assurance case tools that adopt MDE, such as AdvoCATE \cite{denney2017tool}, D-Case Editor \cite{matsuno2010dependability}, ASCE \cite{netkachova2014tool}, Astah GSN \cite{larrucea2017supporting}, and CertWare \cite{barry2011certware}. 
However,  a common issue with these tools is that they all define their own GSN metamodels \cite{denney2017tool}.
This is due to the fact that there has not been an standard GSN metamodel.
Thus, there may be interoperability problems when one wishes to import GSN models created by other tools to his/her own tool. 
Although the tools mentioned above all claim to support SACM, the support was for SACM version 1.0.
Since SACM version 1.0 was not sufficiently explained (no work has been done in this aspect), there may be cognitive gaps between different tool developers, thus the exported SACM models from the tools may differ.
In addition, since SACM version 2.0 was released recently, and there have been a considerable amount of changes, the claimed support for SACM for these tools are out-dated. 

\subsection{Assurance Cases and the Structured Assurance Case Metamodel}
There has been an increasing interest in the use of structured argumentation in other domains, particularly for demonstrating system security \cite{bloomfield2010safety}. 
Such argumentations are typically referred to as \textit{security cases}. 
The similarities between safety and security cases have been highlighted in \cite{lautieri2005safsec}. 
Therefore, the term \textit{assurance case} is a broader definition: \textit{An assurance case should communicate a clear, comprehensible and defensible argument that a system/service is acceptably safe and/or secure to operate in a particular context.} 

To promote standardisation and interoperability, the Object Management Group (OMG) specified and issued the \textit{Structured Assurance Case Metamodel} \cite{sacm}. 
SACM is developed by the developers of system assurance approaches (e.g. GSN and CAE), based on the collective knowledge and experiences of safety and/or security practitioners over the period of two decades. 

In version 2.0 of SACM\footnote{Released in March, 2018}, a considerable amount of changes have been introduced to promote modularity and openness of SACM. 
SACM provides a complete solution for model based system assurance case construction. 
In addition, supporting evidence and related information can also be referenced via reference mechanisms provided in SACM. 
This makes SACM more powerful than existing techniques such as GSN and CAE.

However, in the SACM specification there is limited information on the intended usage of SACM. 
In order to exploit SACM's full potential, and to promote the adoption of SACM, it is necessary to explain SACM in detail so that safety and security engineers can fully use SACM to achieve higher level goals (e.g. automated model-to-text transformation to generate assurance case reports). 

In the current state of practice, graphical notations such as GSN remain the most popular approach for system assurance. 
SACM is designed to support GSN, but the OMG specification does not provide a mapping between GSN elements and SACM elements. 
This is due to the fact there has not been a SACM aligned GSN metamodel. 
Thus, in this paper we provide a GSN metamodel which aligns to SACM. 
There is also a need to translate from GSN to SACM. 
The reason behind this is two fold. First of all, the OMG has not defined a concrete syntax (i.e. graphical notation) for SACM elements, which makes creating SACM models a tedious and error-prone process. 
Thus, to make the transition from GSN to SACM, it is good practice to use GSN notations to construct arguments and then transform from to SACM using model-to-model transformation. Secondly, since GSN has been widely adopted in industry, practitioners can convert their legacy diagrams into GSN models, and then transform to SACM to enable model-based system assurance. 
Due to the reasons above, in this paper, we will provide a mapping (model-to-model transformation) from GSN to SACM.

\subsection{Model-based System Assurance and Cyber-Physical Systems}
%Model based system assurance also brings benefits to system integration at design time. When integrating system components, safety/security engineers can review and integrate assurance cases of the system components. In this scenario, model management operations can be used to achieve (some degree of) automation in this task. This is particularly useful when systems are build using off-the-shelf components with their own assurance cases.

The physical and digital worlds are gradually merging into a largely connected globe. 
This is backed by the emergence of concepts such as Cyber-Physical Systems (CPS).
Openness and adaptivity are core properties of CPS as constituent systems dynamically connect to each other and have to adapt to a changing context at runtime \cite{trapp2013safety}.
CPS harbours the potential for vast economic and societal impact in domains such as automotive, health care, and home automation due to their open and adaptive nature \cite{wei2017deis}.
The majority of application domains of CPS are safety-critical, such as car2car scenarios and collaborative autonomous mobile systems.
If these systems fail, they may cause harm and lead to temporary collapse of important infrastructures, with catastrophic consequences for industry and society.
Therefore, it is imperative to ensure the dependability of CPS in order to realise their full potential. 
However, the open and adaptive nautre of CPS poses significant challenges to assuring such systems, as it is nearly impossible to anticipate the concrete CPS structure, its capabilities and the environmental context sufficiently at design time.

Therefore, existing design time system assurance activities are inappropriate to enable dynamic system assurance for CPS at runtime. 
In \cite{trapp2013safety}, the authors identify the importance of system assurance at runtime for CPS and propose the idea of Models@Runtime and Assurance Case@Runtime, in the sense that system assurance information is exchanged when CPSs interconnect with each other to reason about the dependability of the to-be-formed system of systems.

We envision that SACM is the best candidate for AssuranceCase@Runtime in this context, this is backed by the fact that SACM is used in the DEIS project \cite{wei2017deis} as a backbone for its Open Dependability Exchange metamodel (ODE), to ensure the dependability of CPS. 

%SACM also provides multi-language support, in the sense that argumentations can be expressed using computer languages, which is the premise for automated argument reasoning. This is particularly beneficial for CPSs.
%
%Therefore, model-based assurance cases are more than `nice diagrams', they are important artefacts which can be used at design time for system integration, as well as at runtime for automated reasoning when systems collaborate with each other and form temporary networks. 