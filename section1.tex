\section{Introduction}
Systems/services used to perform critical functions require justifications that they exhibit the necessary properties (e.g. safety and/or security). Assurance cases provide an explicit means for justifying and assessing confidence in these critical properties. In certain industries, typically in the safety-critical domain (e.g. defence, avaiation, automotive and nuclear), it is a regulatory requirement that an assurance case (or a safety case) is developed and reviewed as part of the certification process \cite{healthFound}.
An assurance case is a document that facilitates information exchange between various system stakeholders (e.g. between operator and regulator), where the knowledge related to the safety and security of the system is communicated in a clear and defendable way \cite{}. 

System Assurance is the process of building clear, comprehensive and defensible arguments regarding the properties of systems (i.e. safety and/or security). 
Assurance cases are typically represented textually, using natural languages, or graphically, using structured notations such as the Goal Structuring Notation (GSN) \cite{} or Claims, Arguments and Evidence (CAE) \cite{}. 
The creation of assurance cases is often a manual process, there have been some approaches and tools to apply principals of Model Driven Engineering (MDE) in order to benefit from automation provided by MDE. 
In addition, to improve standardisation and interoperability, the Object Management Group (OMG) also specified and issued a Structured Assurance Case Metamodel (SACM) for the representation of assurance cases. 
Since then, a considerable amount of work has also been done to align GSN and CAE to SACM \cite{}. 

Broadly speaking, the development and acceptance of assurance cases poses significant challenges for engineers and assessors. This is due to the size and complexity of assurance cases, where explicit and implicit dependencies exist between the argumentation for safety/security in the assurance case and its supporting evidence/artefacts, which may include system test data, system failure and hazard analysis, system architecture, etc. In particular, the lack of integration with, and the limited traceability to these evidence/artefacts undermine confidence in the argumentation presented in the assurance case. 

A number of model based system assurance have been proposed, which makes use of MDE to promote automation and the traceability between the assurance case and its supporting artefacts. However, such approaches are ad-hoc approaches which do not fully use the potentials of SACM. Therefore, we speculate that there is a cognitive gap between the intended usage of SACM facilities and the practitioner's understanding of SACM. 

In this paper, we provide a definitive guide to SACM. We provide detailed discussions of the facilities provided by SACM version 2.0. We also provide an official description about the relationship between SACM and the Goal Structuring Notation (GSN), and provide the mapping between SACM and GSN. We also provide an example of how to use SACM to construct an assurance case, and how to maintain the traceability between the argumentation of the assurance case, and its supporting evidence/artefact.

Specifically, the contributions of this paper are as follows:
\begin{itemize}
	\item A definitive guide to SACM version 2.0;
	\item An official mapping between SACM and GSN;
	\item A GSN metamodel that is compliant with SACM;
	\item A detailed discussion on how to use SACM to create a complete system assurance case.
\end{itemize}

This paper is organised as follows...

\section{Background and Motivation}

